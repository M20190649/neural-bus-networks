Bus schedules are unreliable, leaving passengers waiting and increasing commute times.
This problem can be solved by modeling the traffic network, and delivering predicted arrival times to passengers.
Attempts to model traffic networks in research use historical, statistical and learning based models, with learning based models achieving the best results.
This research compares several neural network architectures trained on historical data from Boston buses.
Three models are trained: multilayer perceptron, convolutional neural network and recurrent neural network.
Recurrent neural networks show the best performance when compared to feed forward models.
This indicates that neural time series models are effective at modeling bus networks.
The large amount of data available for training bus network models and the effectiveness of large neural networks at modeling this data show that great progress can be made in improving commutes for passengers.
