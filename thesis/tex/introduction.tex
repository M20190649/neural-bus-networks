\chapter{Introduction}

\section{Traffic and Technology}

- Traffic sucks

Americans spend over 40 hours stuck in traffic a year \cite{traffic}.
This costs the US \$121 billion dollars a year.
Studies from The American Journal of Preventive Medicine have shown that commuting causes a range of negative health side effects \cite{trafficandhealth}.
These effects include raised cholestrol, increased depression risk, increased anxiety, and decreased overall happiness.
Commuting results in lower life satisfaction.
Specifically riding a bus for 30 minutes or longer is connected to the lowest level of life satisfaction compared to other commutes.
All groups including bike commuters experience reduced life satisfaction proportional to the length of the commute.
In addition to the detrimental health effects of, road crashes result in 1.3 million deaths a year \cite{trafficdeaths}.
An additional 20-50 million people are injured or disabled each year in road crashes.
The statistics are worse in low income countries.

- Environmental impact

In addition to health effects, traffic has a strong impact on the environment.
Transportation accounts for 30% of US greenhouse gas emissions.
Road traffic also contributes to reduced air quality, traffic congestion and urban sprawl.

- Self driving cars

Technology can to a great extent solve some of these problems.
Public transit can largely reduce the environmental impact of transporation.
A full bus is 6 times more efficient than a single driver car \cite{trafficenv}.
Additionally, buses emit a tenth of the hydrocarbons compared to single driver cars \cite{trafficenv}.
Recent years have also shown an increase in electric vehicles, which when combined with sustainable energy practices can drastically reduce emissions.

Furthermore, the advent of autonomous cars promises a future of safer roads and vastly fewer road crash deaths.
Interestingly, car ownership shows signs of decline in the US, with millennials waiting longer before buying cars \cite{cars}.
Ride sharing services like Uber are transforming mobility and the auto industry as a whole.
An increase in sensors and connectivity of cars allows large scale optimization which can decrease travel times for users.
All of these technical innovations make it cheaper, safer and more convenient for people to use modern forms of transporation.
However, as mobility becomes available to more people, problems regarding traffic congestion will get worse, not better.

- There will be more traffic

The problem of traffic is also not going anywhere.
Over the next 30 years, the US population is estimated to increase by 70 million \cite{traffic}.
Larger populations, and movement into urban and suburban areas exacerbates the issue.
Traffic in other countries is even worse.
India's transportation system is in crisis with booming population growth in urban areas and increasing vehicle numbers are overwhelming transporation infrastructure.
Clearly, transit technology needs to keep up with exploding demand.

- Coordinating roads and traffic networks is important

One obvious way to reduce congestion is public transportation.
However, the adoption level of public transit systems is very low.
Only about 5% of US working commute to work with public transit \cite{Commute}.
There are several reasons which cause people to prefer private cars to public transit.
Outside of urban areas, public transit can be unavailable or impractical.
Bus riders have to deal with traffic, transfers, and unreliable schedules.
These factors make buses much less convenient than private autos, despite the clear environmental and economic benefits.
This research focuses on improving the reliability of buses for riders by using neural networks to predict bus arrival times.


\section{Bus Network Characteristics}
Describe why it's a hard problem

- Large number of factors

Traffic networks in general are difficult to model.
Part of this has to do with the stochastic nature of bus networks.
Traffic conditions depend on countless variables including weather, driver behavior, time of day, and construction.
Although some of these variables are easy to measure, others are latent.
Furthermore, the relationship between the variables is complicated.
The requires the use of complex models and large amounts of data.

- Describe other bus related features

Besides the sheer number of variables, bus networks also exhibit some behavior which makes them different from typical car networks.
For example, buses trajectories are affected by passenger demand.
The amount of time a bus spends waiting at a stop depends on how many people get off and on at that stop.
Furthermore, a bus may not make all of its stops along a route, and stops may move over time.
Buses also have a schedule to follow, whereas cars do not.
For this reason, some buses will slow down if ahead of schedule to allow other buses to catch up.
These features make bus networks more complex, but they also provide structure for models to learn.

- Clumping

An interesting emergent property of bus networks is clumping.
This term refers to the phenomenon of buses along a route to tend to clump together in groups after starting at uniform intervals.
The effect is caused by the relationship between buses along a route and the respective riders.
Consider the following scenario.
A bus misses a light, and therefore starts running behind schedule.
At future stops, more passengers arrive due to the delay.
The excess of passengers at future stops makes the bus slow down even more, because it takes longer to pick up more passengers.
This produces a positive feedback loop, which causes slow buses to become slower.
Eventually, the previous bus will catch up, causing the buses to clump together.
The opposite happens to a bus when it starts running ahead of schedule, meaning fast buses get faster.
This exacerbates the issue of clumping.
This complex interaction among buses along the same route requires modeling at a route level rather than a bus level.

- High variance

Another issue which arises studying bus networks is the natural randomness of traffic.
Even with a perfect modeling scheme, there are always confounding factors.
This results in a large variance which limits the theoretical best performance for prediction.

\section{Applications}
Vision for the future/Ideal solutions
- What can you do with coordination large networks of cars

